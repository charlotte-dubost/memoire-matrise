\Chapter{ETUDE THÉORIQUE}\label{sec:Theme1}
\begin{table}[htbp]
  \centering
  \caption{Constantes et variables des modèle analytiques}
  \begin{tabular}{|c|l|}
    \hline\rowcolor[gray]{0.8}\color{black}
    Symbole         & Description\\\hline
    $A$           & Aire de la section\\\hline
    $a$       & Rayon exeterne de la roue: $a=R+r_2$\\\hline
    $\delta$             & Flèche imposée par la force de compression\\\hline
    $E$           & Module d'Young du matériau\\\hline
    $E_{p,g}$          & Energie potentielle gravitationnelle\\\hline
    $E_{tot}$          & Energie mécanique totale\\\hline
    $F$             & Force de compression appliquée à la roue\\\hline
    $g$     & Accélération gravitationnelle\\\hline
    $H_{max}$          & Hauteur de saut maximale\\\hline
    $I$           & Moment quadratique de la section\\\hline
    $k$             & raideur équivalente de la roue\\\hline
    $k_1, k_2$           & Rayons de giration principaux de la roue\\\hline
    $m_r$          & Masse de la roue\\\hline
    $\Omega_0$           & Vitesse angulaire à laquelle la roue décrit des cercles (phase 1 du mouvement)\\\hline
    $\omega_0$           & Vitesse de rotation de la roue sur elle même lorsqu'elle roule (phase 1 du mouvement)\\\hline
    $R$       & Rayon médian de la roue\\\hline
    $r_1$ & Rayon interne de section pour une section circulaire\\\hline
    $r_2$             & rayon externe de section pour une section circulaire\\\hline
    $\theta_0$           & Angle d'inclinaison de la roue par rapport à la verticale en régime permanent\\\hline
    $r_c$           & Rayon des cercles obtenus par projection sur le sol du centre de gravité de la roue (phase 1)\\\hline
    $\rho$           & Masse volumique du matériau\\\hline
  \end{tabular}
  \label{tab:Definitions}
\end{table}


\section{Etude du saut d'une roue Cyr}
\subsection{Description du mouvement}
La roue Cyr est comprimée d'une flèche $\delta$ imposée par une force exercée selon l'axe de son diamètre par une force $F$ puis relâchée. Une partie de l'énergie de déformation élastique est transformée en saut. On notera $H_max$ la hauteur correspondant à l'énergie potentielle par laquelle s'élève le centre de gravité de la roue de sa position au repos à sa hauteur maximale.


\subsubsection{Hypothèses}
\begin{itemize}
    \item On considère que le support contre lequel la roue est comprimée est parfaitement rigide.
    \item On néglige la force de trainée de l'air.
\end{itemize}

\subsubsection{Modèle deux degrés de liberté}
On modélise la roue Cyr par deux masses ponctuelles égales $\frac{m_r}{2}$ reliées par un ressort de longueur à vide $2a$, le diamètre externe de la roue, et de rigidité équivalente $k=\frac{4\pi EI}{(\pi^2 -8)R^3}$ \cite{yangkim}. Les positions des masses sont reperées par les ordonnées $y_1$ et $y_2$.

\begin{figure}[htb]
\centering
\includegraphics[width=4in]{images_2ddl/repos.png}
\caption{Système au repos}
\label{fig:repos}
\end{figure}

On note $t_0<0$ le moment auquel on relâche le ressort. $t=0$ correspond au moment auquel la masse inférieure quitte le sol, c'est à dire au moment où la force qui lui est transmise par le ressort devient suffisante pour prendre le dessus sur la gravité:$\frac{m_r g}{2}=k(y_1-2a)$. \\
Les positions des masses à $t=0$ sont donc $y_1=2a+\frac{m_r g}{2k}$ et $y_2=0$
On note $t_f$ le temps pour lequel le centre de gravité du système atteint sa hauteur maximale.

\begin{figure}[htb]
\centering
\includegraphics[width=4in]{images_2ddl/saut.png}
\caption{Saut du système}
\label{fig:saut}
\end{figure}

\subsection{Mise en équations}
\begin{equation}
   t_0<t<0: 
  \begin{cases}
    \frac{m_r}{2}\frac{d^2y_1}{dt^2}+k(y_1-2a)+\frac{m_r}{2}g=0\\
    y_2=0
  \end{cases}
  \label{eq:1}
\end{equation}

On pose: $\tau=t-t_0$
Pour $t_0<t<0$, le mouvement est régi par le système suivant:

\begin{equation}
  \begin{cases}
    \tau=t-t_0 \\
    \alpha_0=\sqrt{\frac{2k}{m_r}}\\
    y_1(\tau)=(\frac{m_r g}{2k}-\frac{F}{k})\cos(\alpha_0 \tau)-\frac{m_r g}{2k}+2a \\
    y_2=0
  \end{cases}
  \label{eq:2}
\end{equation}

\begin{equation}
   0<t<t_f: 
  \begin{cases}
    \frac{m_r}{2}\frac{d^2y_1}{dt^2}+k(y_1-y_2-2a)+\frac{m_r}{2}g=0\\
    \frac{m_r}{2}\frac{d^2y_2}{dt^2}+k(y_2-y_1+2a)+\frac{m_r}{2}g=0
  \end{cases}
  \label{eq:3}
\end{equation}

Avec les conditions initiales: 
\begin{equation}
  \begin{cases}
    y_1(0)=2a+\frac{m_r g}{2k}\\
    \frac{d y1}{dt}(0)=v_{10}\\
    y_2(0)=0 \\
    \frac{d y2}{dt}(0)=0
  \end{cases}
  \label{eq:4}
\end{equation}

Le mouvement pour $0<t<t_f$ est décrit par les équations:

\begin{equation}
  \begin{cases}
    \alpha=\sqrt{\frac{4k}{m_r}}\\
    y_1(t)=\frac{1}{2}(-\frac{g t^2}{2}+v_{10}t+(\frac{m_r g}{2k})\cos{\alpha t}+\frac{v_{10}}{\alpha}\sin{\alpha t}+\frac{m_r g}{2k}+4a) \\
    y_2(t)=\frac{1}{2}(-\frac{g t^2}{2}+v_{10}t-(\frac{m_r g}{2k})\cos{\alpha t}-\frac{v_{10}}{\alpha}\sin{\alpha t}+\frac{m_r g}{2k}) 
  \end{cases}
  \label{eq:5}
\end{equation}

Ce qui nous donne les expressions temporelles de la variation de l'allongement du système et de l'évolution de la positon de son centre de gravité:

\begin{equation}
  \begin{cases}
    y_1(t)-y_2(t)= (\frac{m_r g}{2k})\cos{\alpha t}+\frac{v_{10}}{\alpha}\sin{\alpha t}+2a\\
    \frac{y_1(t)+y_2(t)}{2}=\frac{1}{2}(-\frac{g t^2}{2}+v_{10}t+\frac{m_2 g}{k} +2a)
  \end{cases}
  \label{eq:6}
\end{equation}

Ces équations permettent de tracer des animations du mouvement afin de mieux appréhender ce dernier par la suite. Tout au long de l'élévation du centre de gravité de la roue, on peut observer des déformations correspondant au second mode vibratoire d'un anneau (déformations planes).
\\ 
\\
Bilan d'énergie appliqué au système {masse 1 + masse 2 + ressort}
\begin{itemize}
    \item On prend pour référence de l'énergie potentielle la position du système à l'équilibre statique: $y_{1,ref}=2a-\frac{m_r g}{2k}$ et $y_{2,ref}=0$.
    \item $t<t_0$: $E=\frac{F^2}{2k}$
    \item $t=0$ : $E=\frac{1}{4}m_r v_{10}^2 +\frac{5}{8}\frac{(m_r g)^2}{k} $
\end{itemize}

On en déduit l'expression de la vitesse de la masse 1 à $t=0$: $v_{10}=\sqrt{\frac{2(F^2-\frac{5(m_r g)^2}{4})}{m_r k}}$ \\

Lorsqu'on arrive à $t=t_f$, la vitesse du centre de gravité s'annule. Il suffit alors de dériver l'expression de la position du centre de gravité déterminée ci-dessus pour obtenir: $t_f=\frac{v_{10}}{g}$, soit: 
$$ =\sqrt{\frac{2(F^2-\frac{5(m_r g)^2}{4})}{m_r k g^2}}$$

On substitue ensuite cette expression à $t_f$ dans l'expression de la position du centre de gravité pour svoir la hauteur maximale de ce dernier, à laquelle on soustrait $y_{1,ref}$ pour obtenir finalement l'expression de $H_{max}$.
\\ \\
Ainsi: $H_{max}=\frac{1}{2}(\frac{F^2}{m_r k g}-\frac{0.25m_r g}{k})$
\\
Ou encore, en fonction de la pulsation propre $\alpha$: $H_{max}=\frac{1}{2}(\frac{4F^2}{m_r^2\alpha^2 g}-\frac{g}{\alpha^2})$.
\\ \\
Ainsi, les paramètres influençant la hauteur maximale de saut du système sont:
\begin{itemize}
    \item La force de compression $F$ à laquelle ce dernier est soumis initialement
    \item La masse du système, déterminée par sa géométrie et la masse volumique du matériau
    \item La raideur équivalente du système qui dépend de sa géométrie et du module de Young du matériau.
\end{itemize}

L'expression de $H_{max}$ nous permet d'étudier l'efficacité énergétique du système, c'est à dire d'estimer quelle fraction de l'énergie totale fournie au système sous forme de déformation élastique est convertie en énergie potentielle gravitationnelle servant au saut. \\
Pour cela on s'intéresse au ratio $\frac{E_{p,g}}{E_{tot}}$ des deux quantités décrites précedemment. \\
Avec $E_{tot}=\frac{F^2}{2k}$ et $E_{p,g}=m_r g H_max$, on obtient:  $\frac{E_{p,g}}{E_{tot}}=1-\frac{(m_r g)^2)}{4F^2}$.

\subsection{Traitement numérique du modèle}
Dans la partie qui suit, on étudie pour chaque paramètre dont dépendent $H_max$ et le ratio d'efficacité énergétique comment ces derniers évoluent lorsqu'on fait varier un de ces paramètre indépendamment des autres.
\\ 
\\ 
La variation de ces paramètres doit respecter les valeurs limites suivantes, en dessous desquelles le modèle n'est plus valide:
\begin{itemize}
    \item Pour $F$: Pour que le modèle soit valide il faut qu’il y ait un saut, l'énergie de déformation élastique doit être suffisante pour faire décoller la masse 2, ce qui se traduit par: $\frac{1}{2} \frac{F^2}{k}>\frac{(m_r g)^2}{4k}$  soit: $F>\frac{m_r g}{/sqrt{2}}$
    \item Pour $k$: Le ressort doit pouvoir soutenir la masse 1: $\frac{m_r g}{2k}<2a$ c'est à dire: $k>\frac{m_r g}{4a}$ 
    \item Pour $m_r$: Diminuer $m_r$ revient à retirer de la matière, il y a donc une masse limite dépendant des propriétés mécaniques du matériau en dessous de laquelle il y aura rupture lors de la compression, la roue étant devenue trop fragile.
\end{itemize}
Les limites seront indiquées par des pointillés sur les tracés plus bas
\\
On applique le modèle développé dans la section précédente aux quatre cas suivants:
\begin{itemize}
	\item Une roue en acier dont les dimensions ont été mesurées sur une roue existante (cas témoin)
	\item Une roue en matériau composite de dimensions identiques à celles de la première
	\item Une roue en matériau composite de section elliptique dont le prolongement du petit axe coincide avec l'axe de révolution (cas 1)
	\item Une roue en matériau composite de section elliptique dont le prolongement du grand axe coincide avec l'axe de révolution (cas 2)
\end{itemize}
\\
\\

figures: ellipse cas 1 et 2 + geo section circulaire


Si on fait varier un paramètre indépendamment des autres, il est nécessaire d'affecter une valeur à ces derniers, fixés momentanément:
\begin{itemize}
	\item Les géométries considérées sont celles présentées en figure_
	\item On considère que la force $F$ appliquée au système a pour norme $900 N$, correspondant au poids d'un athlète se suspendant à une main à la roue.
	\item Les propriétés mécaniques de l'acier sont les suivantes $E_a=210GPa$, $rho_a=7800 kg m^-3$
	\item Pour le matériau composite, on considère dans un premier temps qu'il s'agit d'une matrice nylon renforcée de fibres de carbones courtes, qu'on considère isotrope. On a donc: $E_c=9 GPa$, $G=3.8 GPa$, $rho_c=1170 kg m^-3$, avec une contrainte à la rupture $\sigma_r=63 MPa$

Dans ce qui suit, lorsqu'on fait varier une seul paramètre, les autres, momentanément fixés, prennent les valeurs qui leur ont été attribuées ci-dessus.
\\
\\ 
Avec ces données il est possible de déterminer la masse minimale que devra avoir la roue (on considère le cas d'une section circulaire):
\begin{itemize}
\item On commence par calculer les contraintes lorsque la roue est comprimée par une force $F$: La contrainte dans l'arrête extérieure s'exprime $\sigma_i=k_i\sigma$ \cite{roark},
où sigma est la contrainte calculée pour une poutre droite: $\sigma=\frac{Mr_2}{I_z}$, avec $I_z=\frac{\pi}{2}(r_2^4-r_1^4)$ et $M=\frac{\pi}{4}FR$.
D'après les formules du livre de Roark, $k_i$ s'exprime $k_i=\frac{1}{4\beta}\frac{1-\beta}{\frac{R_c}{r_2}-1}[1+(\frac{r_1}{r_2})^2]$ avec $\beta=\frac{1}{2}[\frac{2R_c}{r_2}-\sqrt{(\frac{R_c}{r_2})^2-1}-\sqrt{(\frac{R_c}{r_2})^2-(\frac{r_1}{r_2})^2}]$ \\
Il ne reste plus qu'à déterminer le rayon de courbe de la roue comprimée, comme illustré sur la figure ci-dessous: $R_c=\frac{a_r^2}{b_r}$, $a_r$ et $b_r$ étant respectivement les demi grand axe et demi petit axe de l'ellipse que devient la roue lorsqu'elle est comprimée.\\
Pour calculer $a_r$ et $b_r$, on utilise les formules du livre de Roark qui donnent les variations de diamètre horizontal $\DeltaD_h$ et vertical $\DeltaD_v$ d'une anneau comprimé selon son diamètre avec une force $F$, et on aura ainsi $a_r=R+\frac{\DeltaD_h}{2}$ et $b_r=R+\frac{\DeltaD_v}{2}$.\\
D'après les formules de Roark, $\DeltaD_h=\frac{FR^3}{EI}(\frac{1}{2}(1+\frac{I}{AR^2}+\frac{2EI}{GAR^2})-1+\frac{2}{/pi}(1-\frac{I}{AR^2})^2)$ et $\DeltaD_v=-\frac{FR^3}{EI}(\frac{\pi}{4}(1-\frac{I}{AR^2}+\frac{2EI}{GAR^2})-\frac{2}{/pi}(1-\frac{I}{AR^2})^2)$, où $A$ est l'aire de section, $E$, le module de Young, $G$, le module de cisaillement, $R$, le rayon médian de la roue, et $I$ le moment de section quadratique selon l'axe principal perpendiculaire au plan de la roue.
\item En implémentant numériquement les équations ci-dessus, on est capable de calculer les contraintes maximales auxquelles sera soumise la roue pour une force de compression $F$ donnée. Pour $F=900N$, en fixant $r_2$ et en faisant varier $r_1$ de $0$ à $r_2$, on est capable de tracer les contraintes maximales, en fonction de la masse de la roue, $m_r=2\pi\rho A R$. Comme l'illustre la figure ci-dessous, il est alors possible de déterminer la masse à partir de laquelle les contraintes passent sous le seuil de $63 MPa$, et donc à partir de laquelle il y a rupture.

figure: stress/masse: Contraintes maximales dans la roue en fonction de sa masse, pour $r_2=2.5 cm$ et $r_1$ variant de $0.0$ à $2.49 cm$ (gauche) et $r_2=5.0 cm$ et $r_1$ variant de $0.0$ à $4.99 cm$


\item On peut donc implémenter ce procédé numériquement pour déterminer cette valeur minimale de la masse pour en fonction de $r_2$. Le resultat de l'implémentation est présenté à la figure ci-dessous. 

figure: Masse minimale de la roue en fonction de $r_2$


\item Pour ce qui suit, $r_2$ sera fixé à $1.75 cm$, la masse minimale sera donc $2.0 kg$

\\
\\
Variations de $H_{max}$ et de $\frac{E_{p,g}}{E{tot}}$

Figures

Remarques:
\begin{itemize}
\item Les tracés adimensionnels permettent d'étudier une tendance globale, ainsi que de s'affranchir de la dépendance des résultats avec les autres paramètres.
\item Le rendement énergétique minimal est 0.57, lorsque $F$ est égale à sa valeur limite
\item Pour les roues composites définies plus haut, et le modèle étudié présentement, le rendement énergétique avoisine les 100\% à partir d'une certaine force ($F=200 N$ dans notre cas)
\item Une variation de la masse influe peu sur l'efficacité énergétique, qui reste élevée sur la figure_, tracée pour une force $F=900 N$. La force appliquée à la roue est le paramètre déterminant pour cette dernière. Dans la continuité de cette observation, sur la figure_ (Hmaxf) l'accentuation de la pente, croissante avec la force, traduit une "rentabilité" de l'énergie d'autant meilleure que la force qu'on applique est élevée, surtout pour la roue composite à section circulaire.
\item A l'inverse, la courbe $H_{max}=f(k)$ s'aplatit rapidement à mesure que la raideur augmente. Il s'agit de trouver un compromis entre la solidité et la maximisation de la hauteur de saut.












































