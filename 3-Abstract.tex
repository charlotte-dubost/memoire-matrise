% Abstract
%
% Résumé de la recherche écrit en anglais sans être
% une traduction mot à mot du résumé écrit en français.

\chapter*{ABSTRACT}\thispagestyle{headings}
\addcontentsline{toc}{compteur}{ABSTRACT}
%

The Cyr Wheel is a circus apparatus consisting of a 2 meters diameter (size may vary with the height of the user) ring made of steel or aluminum, with a weight varying between 10kg and 20kg. The user stands inside the wheel and performs acrobatic tricks while spinning. This utilization makes the weight, rigidity and geometry determinant influence parameters for the way the user will perform on the wheel. Since its invention in 1996 it has been revisited several times by circus artists and manufacturers, and especially with the utilization of advanced materials for sport equipment spreading to circus structures (such as poles and gantries), the circus community shows interest for the possibilities that a lighter, more flexible Cyr wheel could bring (like jumping and bouncing with the wheel, for instance). The fact that a Cyr wheel is made of curved elements leads to a significant increase in terms of fabrication costs, and none of the Cyr wheel projects involving new composite materials has been completed to this day, mostly because of a lack of means, and the fact that it is too expensive to get to the right prototype by the ‘trial and error’ method. On the other hand, developing a theoretical model of the dynamics of Cyr wheel is a complex task which requires an in-depth study since its motion can be linked to the Euler’s disk model. 
For now, advances on this question consist in trials of prototypes manufactured with no prior theoretical studies which resulted in a lack of optimization and unsuitable material choices leading to failure when tested.\\ 

To this day there aren’t any scientific publications specific to the Cyr wheel dynamics or the use of composite materials for circus structures. Nevertheless, Euler’s disk and spinning coins have been widely studied by the scientific community due to the chaotic aspect of their motion in its terminal stage as well as the influence of friction and air viscosity on its behavior. The mathematical model developed to study this problem provide a basis that will be adapted the dimension, shape and use of the Cyr wheel. The ability to jump with the Cyr wheel being a determinant aspect of the study, the model developed by Yang and Kim about energy storage in elastic rings will also be used as a basis an adapted to the scale and problematics specific to a Cyr wheel.\\

Carrying out the study of a composite Cyr wheel from its design to the test and validation of a prototype requires to combine at the same time practical knowledge of the apparatus, theoretical knowledge about composite materials, 3D printing processes and dynamics of Euler’s disk-like motions, as well as mastery of numerical modelling and large scale composite 3D printing equipment. These conditions are hard to reunite at once, which explains why, contrary to other circus apparatuses, the question of the use of composite materials for Cyr wheel manufacturing has not been solved yet and why no scientific publication has been released on this subject. Answering this problematic represents a significant evolution of the discipline of Cyr wheel and of the manufacturing of this apparatus. On top of that, it will allow to bring quantitative answers to multiple questions that have only be tackled intuitively until then.\\
The general objective of this study is to estimate the mechanical properties and geometry that will allow to optimize and diversify the use of a Cyr wheel, choose a corresponding material to 3D print the Cyr wheel in, print a real size prototype and experiment on the new acrobatic possibilities of this new apparatus.

The guidelines of the project are the following research questions:
\begin{itemize}
\item What would be the optimal material properties (Young modulus, density) and geometry that would expand the acrobatic possibilities of a Cyr wheel ? 
\item	How would this circus discipline evolve with a corresponding Cyr wheel made out of a high-performance optimized material ? 
\end{itemize}


The project is based on the following hypothesis:
\newline
Developing a Cyr wheel with a rigidity and a weight different from the existing ones will significantly contribute to the evolution of the discipline.\\
This hypothesis will be refuted if the new acrobatic tricks that we suppose this new type of Cyr wheel will allow are too demanding (physically or in terms of skills) and can’t be achieved by a professional Cyr wheel performer, even with some practice time.\\

Justification of the originality: Considering the dynamics of a spinning Cyr wheel with a performer, adding elastic properties and a lighter weight sounds promising and has been tried several times, but never achieved.\\
Success criteria: a successful contribution would consist in advances in the theoretical knowledge of Cyr wheel, by quantitatively answering problematics raised by the circus community, evolution of the discipline by expanding the acrobatic possibilities of this apparatus and providing answers concerning the interest of the introduction of composite materials for Cyr wheels to manufacturers.\\

The project will be carried out with the following methodology:

\begin{enumerate}
\item Theoretical study:
develop a mathematical model of Cyr wheel motions and of energy storage, jumping and bouncing. Once the model is established, identify the parameters of significant influence among material properties and geometry parameters (Young modulus, density, moment of inertia, size of diameter…). Study the scalability and develop an dimensionless analysis.

\item Numerical study:
develop a numerical model of Cyr wheel motions, study the dynamic stability and energy storage capability. Quantify the influence of the identified parameters by studying the impact of their variation on the motion.

\item Proof of concept:
once the optimal properties of the material are estimated thanks to the two previous phases, develop a corresponding composite material and 3D print a small part of the wheel with the same radius of curvature than the real-size one. Once it is tested and validated, print a small size prototype and test it applying appropriate strength estimated with the scalability study.

\item Real size prototyping:
3D-print the final prototype: a real size Cyr wheel made out of the optimized high performance composite material previously defined

\item Test of the prototype:
validate that the chosen material is appropriate.
Explore the news acrobatic possibilities provided by the prototype.

\end{enumerate}


The impacts and benefits of the project are the following:
\begin{itemize}
\item Carrying out this study will allow circus equipment manufacturers to further the reflection and conclude on the pros and cons of adding composite Cyr will to their current offer.
\item Developing a numerical model of Cyr wheel motions will allow to quantify the influence of different parameters such as rigidity, density and geometry on the performance of the user. This will bring an answer to problematics left unanswered until now such as the ones put forward by the research department of Montreal National Circus School (ideal user’s weight/ Cyr wheel weight ratio, ideal user’s height, Cyr wheel diameter ratio) 
\item The resulting research article will provide a basis for further investigation about the Cyr wheel, since no publication has been released on this topic to this day.
\item On a broader perspective, this study will be a contribution to advances in the research on sporting equipment involving circular shapes and problematics of lightness and flexibility.

\end{itemize}
